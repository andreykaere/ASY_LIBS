\documentclass[12pt]{extarticle}




%\usepackage{tgtermes}

\usepackage{amsthm}
\usepackage{thmtools}
\usepackage[T1]{fontenc}

\usepackage{graphicx}
\usepackage{float}
\usepackage[margin=0.7in]{geometry}
\usepackage{caption}
\usepackage{csquotes}
\usepackage[export]{adjustbox}
\usepackage{wrapfig}
\usepackage{setspace}
\usepackage{anyfontsize}
\usepackage{titlesec}

\usepackage{lmodern}


\usepackage{tcolorbox}

\newenvironment{codeleft} 
    %{\begin{flushleft}\fontfamily{qcs}\fontsize{11}{11}\selectfont}
    {\begin{flushleft}\ttfamily \fontseries{m}\selectfont\tt}
    {\end{flushleft}\par\noindent}

\newenvironment{codecenter} 
    {\begin{center}\bf}
    {
    \end{center} \\
    }


\def\inc#1{{\ttfamily \fontseries{m}\selectfont\texttt{#1}}}
%    {{\fontfamily{qcs}\fontsize{11}{11}\selectfont #1}}

\newtcolorbox{mybox}[3][]
{
  colframe = #2!25,
  colback  = #2!10,
  coltitle = #2!20!black,  
  title    = {#3},
  #1,
}

%\def\commentcode#1#2
%   {\indent\inc{#1}\indent \qquad \qquad #2}

%\newcommand{\commentcode}[2]{
%   \begin{tcolorbox}{\inc{#1}\\ } \newline \vspace{-8.5mm} \newline {\null \qquad \qquad #2} \end{tcolorbox}}

\def\commentcode[#1]#2#3{
  \begin{mybox}{#1}{} {\inc{#2}\\ } \newline \vspace{-8.5mm} 
\newline {\null \qquad \qquad #3} \end{mybox}}

%\newcommand{\q}{\textquote}
\let\q\textquote

\newcommand{\struct}[1]
    {{\ttfamily \fontseries{m}\selectfont\texttt{#1}} }


\begin{document}

\tableofcontents
\newpage

\section{Introduction}

Before using this package, make sure, that you have this settings:
\begin{codeleft}
	settings.outformat = "pdf";\\
	settings.render = 0;\\
	settings.prc = false;\\
\end{codeleft}
and specified size of picture by \inc{size3}. Also, you have to
wrap your code into function (say \inc{main}) and put 
\inc{with\_geometry3d(main);} after \inc{main} 
function ends.
%in the very end of your program

\subsection{Objects types list}

The package \textit{geometry3d.asy} is the extension of the module 
\textit{geometry.asy}. Basically, this package provides you a tools 
to creare a really nice 3D pictures in solid geometry. 

Here is all types, defined in this module\\
\inc{basis3} -- a 3D ray \\
\inc{curve3} -- a 3D ray \\
\inc{ray3} -- a 3D ray \\
\inc{vector3} -- a 3D vector \\
\inc{line3} -- a 3D line \\
\inc{planeLine3} -- a finite line on the given plane \\
\inc{circle3} -- a 3D circle \\
\inc{plane3} -- a plane \\
\inc{sphere3} -- a sphere \\
%\textbf{}



\section{Temp: all functions}


\commentcode[green]{void drawAllObjects();}{this function draws all objects on the scene with front-back feature and is called by default in function 
\inc{with\_geometry3d}. } 


\commentcode[green]{void withGeometry3d(void main());}{this function 
is meant to be ending of your programm, executing essential function for 
drawing figures properly.}

\commentcode[green]{void add2dFrame();}{add 2D frame in order to be able 
to draw a 2D figures}

\commentcode[green]{void drawCurve(picture pic=currentpicture, 
curve3 curve, pen frontpen=currentpen, 
pen backpen=currentpen+dashed);}{draw \inc{curve} with pens 
\inc{frontpen} and \inc{backpen} respectively. }

\commentcode[red]{circle3 circle3(triple A, triple B, triple C);}{
returns circumcircle of triangle \(ABC\).}

\commentcode[red]{circle3 incircle3(triple A, triple B, triple C);}{
returns incircle of triangle \(ABC\).}


\commentcode[red]{transform3 orthogonalproject(plane3 p);}{
returns \inc{transform3}, which projects in direction of normal to the 
\inc{plane p}. }


\commentcode[red]{triple foot3(triple A, line3 l);}{
return the foot of the perpendicular dropped from point \(A\) onto the 
line \(l\). }

\commentcode[red]{triple foot3(triple A, plane3 p);}{
return the foot of the perpendicular dropped from point \(A\) to the 
plane \(p\). }

\commentcode[green]{void markrightangle3(triple A, triple B, triple C, real n=5, pen p=currentpen);}{
marks right angle \(\angle ABC\) with \inc{pen p}, size of \inc{real n}. }


\commentcode[green]{real distance3(triple A, triple B);}{
returns distance between two points \(A\) and \(B\).}

\commentcode[green]{triple midpoint3(triple A, triple B);}{
returns the midpoint of segment \(AB\).}

\commentcode[green]{basis3 get\_basis(projection P = currentprojection);}{
returns the basis of the \inc{projection P} formed from vectors 
\(\vec{x}=\) \inc{P.camera}, \(\vec{y}=\) \(\vec{x} \times \vec{u}\),
\(\vec{z}=\) \(\vec{x} \times \vec{y}\), where \(\vec{u} =\) 
\inc{P.up}. }

%\(\vec{x}=\) \inc{P.camera}, \(\vec{y}=\) \inc{cross(P.camera, P.up)},
% \(\vec{z}=\) \inc{cross(\(\vec{x}, \vec{y}\))}. }

\commentcode[green]{triple calcCoordsInBasis(basis3 basis, triple A);}{
returns coordinates of point \(A\) (which coordinates are given 
in standart basis \(\{\vec{x}, \vec{y}, \vec{z}\}\)) in basis \inc{basis}. }


\commentcode[green]{triple changeBasis(basis3 basis1, basis3 basis2, 
triple A);}{
returns coordinates of point \(A\) (which coordinates are given 
in basis \inc{basis1}) in basis \inc{basis2}. }

\commentcode[green]{pair project3(triple A);}{
returns 2D-coordinates \((x',y')\) of \inc{triple A} as if it was drawn
as a plain point \(A'\) with coordinates \((x',y')\). \\
\newline
\texttt{WARNING!} It won't work unless you specified size of image with 
\inc{size3}. }

\commentcode[green]{path project3(path3 p);}{
returns 2D-path formed from \inc{project3(node)} for each node of nodes 
of \inc{path3 p}.
}
\commentcode[green]{void markangle3(picture pic = currentpicture,
               Label L = "", int n = 1, real radius = 0, real space = 0,
               explicit triple A, explicit triple B, explicit triple C,
               pair align = dir(1),
               arrowbar3 arrow3 = None, pen p = currentpen,
               filltype filltype\_ = NoFill,
               margin margin = NoMargin, marker marker = nomarker);}{
marks angle \(\angle ABC\) with \inc{pen p}, filled with 
\inc{filltype\_}, drawing arrow with \inc{arrow3}.
}
%\commentcode[green]{


\commentcode[red]{bool collinear3(triple A, triple B, triple C);}{
returns \inc{true} if points \(A,B,C\) are collinear, otherwise 
it will return \inc{false}.
}
\commentcode[blue]{circle3 Circle(triple C, triple A, triple normal=Z);}{
returns circle with center at \(C\) and normal \inc{normal}, passing 
through point \(A\).
}

\commentcode[green]{line3 parallel(line3 a, triple A);}{
returns line, which is parallel to given line \(a\) and passing through
point \(A\).
}

\commentcode[red]{bool isIntersecting(line3 a, plane3 s, bool inf=true);}{
returns \inc{true} if line \(a\) and plane \(s\) intersect, otherwise --
\inc{false}. If \inc{inf=false}, then plane \(s\) is not considered 
infinite.
}

\commentcode[green]{line3 invertpoint(pair A, projection 
P=currentprojection);}{
returns line, which contains point \(A\) and has the vector 
\inc{P.camera} as its direction vector. This function will 
be still working if \(A\) has a type \inc{triple}.
}


\commentcode[green]{triple getpointX(real x, line3 a);}{
For the type \inc{line3} are available functions 
\inc{getpointX, getpointY, getpointZ}, which calculate the rest of the 
coordinates of the point on the given line by one given coordinate. }
    
\commentcode[green]{triple getpointXY(real x, real y, plane3 a);}{
For the type \inc{plane3} are defined analogical functions 
\inc{getpointXY, getpointYZ, getpointXZ}, 
which calculate the last one of the coordinates of the point on the 
given plane by two given coordinates. }

\commentcode[green]{bool isIntersecting(line3 a, line3 b);}{
returns \inc{true} if lines \(a\) and \(b\) intersect.
}

\commentcode[green]{bool isSkew(line3 a, line3 b);}{
returns \inc{true} if lines \(a\) and \(b\) are skew.
}

\commentcode[green]{bool isParallel(line3 a, line3 b);}{
returns \inc{true} if lines \(a\) and \(b\) are parallel.
}


\commentcode[green]{triple intersectionpoint(line3 a, line3 b);}{
returns the intersection point of the lines \(a\) and \(b\).
}

%\commentcode[red]{














\section{The type \struct{line3}}



\section{The type \struct{sphere3}}

Represent sphere \inc{sphere(C,r);} as a circle \inc{
Circle(project3(C),r);} from package \inc{graph}.
%\commentcode{void drawline(picture pic=currentpicture, pair P, pair Q, pen p=currentpen);}{draw the visible portion of the (infinite) line going through P and Q, without
%altering the size of picture pic, using pen p.}


\end{document}
