\documentclass[12pt]{extarticle}




%\usepackage{tgtermes}

\usepackage{amsthm}
\usepackage{thmtools}
\usepackage[T1]{fontenc}

\usepackage{graphicx}
\usepackage{float}
\usepackage[margin=0.7in]{geometry}
\usepackage{caption}
\usepackage{csquotes}
\usepackage[export]{adjustbox}
\usepackage{wrapfig}
\usepackage{setspace}
\usepackage{anyfontsize}
\usepackage{titlesec}

\usepackage{lmodern}


\usepackage{tcolorbox}

\newenvironment{codeleft} 
    %{\begin{flushleft}\fontfamily{qcs}\fontsize{11}{11}\selectfont}
    {\begin{flushleft}\ttfamily \fontseries{m}\selectfont\tt}
    {\end{flushleft}\par\noindent}

\newenvironment{codecenter} 
    {\begin{center}\bf}
    {
    \end{center} \\
    }


\def\inc#1
    {{\ttfamily \fontseries{m}\selectfont\texttt{#1}} }
%    {{\fontfamily{qcs}\fontsize{11}{11}\selectfont #1}}


%\def\commentcode#1#2
%   {\indent\inc{#1}\indent \qquad \qquad #2}

\newcommand{\commentcode}[2]{
   \begin{tcolorbox}{\inc{#1}\\ } \newline \vspace{-8.5mm} \newline {\null \qquad \qquad #2} \end{tcolorbox}}

%\newcommand{\q}{\textquote}
\let\q\textquote

\newcommand{\struct}[1]
    {{\ttfamily \fontseries{m}\selectfont\texttt{#1}} }


\begin{document}

\tableofcontents
\newpage

\section{Introduction}

Before using this package, make sure, that you have this settings:
\begin{codeleft}
	settings.outformat = "pdf";\\
	settings.render = 0;\\
	settings.prc = false;\\
\end{codeleft}
and specified size of picture by \inc{size3}. Also, you have to
wrap your code into function (say \inc{main}) and put 
\inc{with\_geometry3d(main);} after \inc{main} 
function ends.
%in the very end of your program

\subsection{Objects types list}

The package \textit{geometry3d.asy} is the extension of the module 
\textit{geometry.asy}. Basically, this package provides you a tools 
to creare a really nice 3D pictures in solid geometry. 

Here is all types, defined in this module\\
\inc{basis3} -- a 3D ray \\
\inc{curve3} -- a 3D ray \\
\inc{ray3} -- a 3D ray \\
\inc{vector3} -- a 3D vector \\
\inc{line3} -- a 3D line \\
\inc{plane3} -- a plane \\
\inc{sphere3} -- a sphere \\
%\textbf{}



\section{Temp: all functions}


\commentcode{void drawAllObjects();}{this function draws all objects on the scene with front-back feature and is called by default in function 
\inc{with\_geometry3d}. } 


\commentcode{void withGeometry3d(void main());}{this function 
is meant to be ending of your programm, executing essential function for 
drawing figures properly.}

\commentcode{void add2dFrame();}{add 2D frame in order to be able 
to draw a 2D figures}

\commentcode{void drawCurve(picture pic=currentpicture, curve3 curve, 
pen frontpen=currentpen, pen backpen=currentpen+dashed);}{draw 
\inc{curve} with pens \inc{frontpen} and \inc{backpen} respectively. }

\commentcode{circle3 circle3(triple A, triple B, triple C);}{
returns circumcircle of triangle \(ABC\).}

\commentcode{circle3 incircle3(triple A, triple B, triple C);}{
returns incircle of triangle \(ABC\).}


\commentcode{transform3 orthogonalproject(plane3 p);}{
returns \inc{transform3}, which projects in direction of normal to the 
\inc{plane p}. }


\commentcode{triple foot3(triple A, line3 l);}{
return the foot of the perpendicular dropped from \inc{triple A} onto the 
\inc{line3 l}. }

\commentcode{triple foot3(triple A, plane3 p);}{
return the foot of the perpendicular dropped from \inc{triple A} to the 
\inc{plane3 p}. }

\commentcode{void markrightangle3(triple A, triple B, triple C, real n=5, pen p=currentpen);}{
marks right angle \(\angle ABC\) with \inc{pen p}, size of \inc{real n}. }

\commentcode{}{}
\commentcode{}{}
\commentcode{}{}
















\section{The type \struct{line3}}



\section{The type \struct{sphere3}}

Represent sphere \inc{sphere(C,r);} as a circle \inc{
Circle(project3(C),r);} from package \inc{graph}.
%\commentcode{void drawline(picture pic=currentpicture, pair P, pair Q, pen p=currentpen);}{draw the visible portion of the (infinite) line going through P and Q, without
%altering the size of picture pic, using pen p.}


\end{document}
