\documentclass[12pt]{extarticle}

%\usepackage{tgtermes}

\usepackage{amsthm}
\usepackage{thmtools}
\usepackage[T1]{fontenc}

\newenvironment{codeleft} 
    {\begin{flushleft}\fontfamily{qcs}\fontsize{11}{11}\selectfont}
    {\end{flushleft}\par\noindent}

\newenvironment{codecenter} 
    {\begin{center}\fontfamily{put}\selectfont}
    {
    \end{center} \\
    }


\def\inlinecode#1
    {{\fontfamily{qcs}\fontsize{11}{11}\selectfont #1}}


\def\commentcode#1#2
    {\inlinecode{#1}\\ \qquad #2}


\begin{document}

\tableofcontents
\newpage

\section{Introduction}

Before using this package, make sure, that you have this settings:\\
\begin{codeleft}
	settings.outformat = "pdf";\\
	settings.render = 0;\\
	settings.prc = false;\\
\end{codeleft}
and specified size of picture by \inlinecode{size3}.

\subsection{Objects types list}

The package \textit{geometry3d.asy} is the extension of the module 
\textit{geometry.asy}. Basically, this package provides you a tools 
to creare a really nice 3D pictures in solid geometry. 

Here is all types, defined in this module\\
\inlinecode{ray3} -- a 3D ray \\
\inlinecode{vector3} -- a 3D vector \\
\inlinecode{line3} -- a 3D line \\
\inlinecode{plane3} -- a plane \\
\inlinecode{sphere3} -- a sphere \\
%\textbf{}


\section{The type "line3"}

\commentcode{void drawline(picture pic=currentpicture, pair P, pair Q, pen p=currentpen);}{
draw the visible portion of the (infinite) line going through P and Q, without
altering the size of picture pic, using pen p.}


\end{document}
